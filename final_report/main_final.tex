\documentclass[11pt]{article}
\usepackage{fullpage,graphicx,psfrag,amsmath,amsfonts,verbatim}
\usepackage[small,bf]{caption}
\usepackage{pstool}
\graphicspath{{figs/}}
\let\epsilon\varepsilon
\input defs.tex
\newcommand{\proj}{\ensuremath{\Pi_{\B(R)}}}
\newcommand\numberthis{\addtocounter{equation}{1}\tag{\theequation}}
\bibliographystyle{alpha}

\title{Projected Gradient Descent Efficiently\footnote{Technically, a conjecture.}~ Solves the Trust Region Subproblem}
\author{Mark Nishimura \and Reese Pathak}

\begin{document}
\maketitle

\begin{abstract}
We show that projected gradient descent asymptotically converges to a global minimizer of 
the trust region subproblem.
We remark on next steps in this project at the end.
\end{abstract}

% \newpage
% \tableofcontents
% \newpage

\section{Introduction}
Trust region methods are sequential programming procedures in which heuristics are used to approximately solve a general optimization problem through multiple constrained quadratic programs. As a subroutine, these methods formulate and solve many instances of the following \emph{trust region subproblem}
\begin{equation}\label{problem:TR}
\begin{array}{ll} 
\mbox{minimize} & (1/2)x^TAx + b^T x \\
\mbox{subject to} & 
\|x\| \leq R\\
\end{array}
\end{equation}
with variable $x \in \R^n$. 
The problem data are a symmetric matrix $A \in \R^{n \times n}$, a vector $b \in \R^n$, and a radius parameter $R > 0$. Crucially, the matrix $A$ is possibly indefinite. 
\subsection{Previous works}
The trust region subproblem is well-studied, and thus there many previous works worth mentioning. In earlier papers, the problem was solved either via subspace methods such as Steihaug-Toint (where no global convergence guarantees have been proven, to our knowledge), or using fast eigenvector and eigenvalue computation procedures like the Lanczos method \cite{conn2000, erway2009, gould1999, gould2010}. More recently, however, some authors have provided convergence guarantees for this problem. For example, by reducing the trust region subproblem to a sequence of approximate eigenvector computations, Hazan and Koren \cite{hazan2016}
demonstrate that $\tilde O(1/\sqrt{\epsilon})$\footnote{We use the $\tilde O(\cdot)$ notation to hide logarithmic factors.} matrix-vector multiplies are enough to guarantee an 
$\epsilon$-suboptimal point. In \cite{nguyen2017}, Nguyen and Kilin{\c{c}}{-}Karzan reduce the trust region problem to a convex QCQP using eigenvector calculations, where first-order methods apply. 

However, perhaps the most obvious algorithm to solve \eqref{problem:TR}, is the
\emph{projected gradient method}, which we study in this paper. To our knowledge, the only previous work 
that analyzes the convergence properties of this procedure on 
\eqref{problem:TR} is \cite{tao1998}, where Tao and An augment this procedure by a restarting scheme, requiring 
possibly $O(d)$ restarts, which could scale poorly for large-scale problems. 
We also mention a recent work by Carmon and Duchi \cite{carmon2016}, studying the closely related problem
\begin{equation}\label{problem:QP}
\text{minimize}~ (1/2)x^T A x + b^Tx + (\rho/3) \|x\|_2^3,
\end{equation}
in variable $x \in \R^n$, again with $A$ symmetric, possibly indefinite, and parameter $\rho > 0$. The 
authors analyze gradient descent, proving that $\tilde O(1/\epsilon)$ gradient steps are enough to output an $\epsilon$-suboptimal point.

In this paper we demonstrate that the projected gradient method on \eqref{problem:TR} 
asymptotically converges to a global minimizer on the trust region subproblem.

%% In \S\ref{sec:prelims} 
%% we prove that projected gradient descent is a descent method, in particular, converging to 
%% the global minimizer of the objective in problem \eqref{problem:TR}. 

\subsection{Notation and classical results}
In the sequel, we refer to the objective function as $f: \R^n \to \R$, given by $f(x) = (1/2)x^T Ax + 2b^T x$.
Additionally, the constraint set is the closed ball
$\B(R) \triangleq \{x \in \R^n \mid \|x\| \leq R\}$, 
where $\|\cdot\|$ denotes the Euclidean norm. We use the notation $x^\star$ to denote the global minimum of $f$ when it is unique,
so that $x^\star = \argmin_{x \in \B(R)} f(x)$. We use $f^\star$ to denote the optimal value of $f$, so that
$f^\star = \inf_{x \in \B(R)} f(x)$. Hence, when $x^\star$ exists, $f^\star = f(x^\star)$. 

We fix the eigendecomposition of $A = UDU^T$, where $D = \diag(\lambda_1, \dots, \lambda_n)$, and $U$ has orthonormal
columns $u_i$. We impose without loss that $\lambda_1 \leq \lambda_2 \leq \cdots \leq \lambda_n$. 
By $\|\cdot\|_\mathrm{op}$, we denote the $\ell_2$-operator norm 
$\|M\|_\mathrm{op} = \sup_{\|x\| = 1} 
\|Mx\|$, for any $M \in \R^{n \times n}$. 
A useful identity is that 
$\|M\|_\mathrm{op} = \max_i |\lambda_i(M)|$ when 
$M$ is a symmetric $n \times n$ matrix. We will put $\beta \triangleq \|A\|_{\mathrm{op}}$.

Additionally, say a differentiable function $g: \R^n \to \R$ is 
$L$-smooth on convex set $C \subset \R^n$, provided that 
\[
\|\nabla g(x) - \nabla g(y)\| \leq L \|x - y\| \qquad 
\text{for any $x, y \in C$}.
\]
It is well known that this implies 
\begin{equation}\label{ineq:smoothness}
g(x) - g(y) \leq  \nabla g(y)^T(x - y) + \frac{L}{2}\|x -y\|^2 \qquad 
\text{for any $x, y \in C$}.
\end{equation}
Equivalently, $\|g(x)\|_\mathrm{op} \leq L$, for Lebesgue almost every $x \in C$. For nonempty, closed, convex sets $C 
\subset \R^n$, associate the projection operator $\Pi_C: \R^n \to C$ given by 
\[
\Pi_C(x) = \argmin_{y \in C} \left( \frac12 \|x - y\|^2 \right), 
\]
for any $x \in \R^n$. In the sequel we denote by $I: \R^n \to \R^n$ the identity operator on $\R^n$.

\section{Asymptotic convergence to a global minimizer}\label{sec:prelims}

\subsection{Projected gradient descent}
Projected gradient descent (PGD) begins at an initialization $x^{(0)} \in \R^n$ and generates iterates
\begin{align*}
    y^{(k + 1)} &= x^{(k)} - \eta \nabla f(x^{(k)}) 
    %= (I - 2\eta A)x^{(k-1)} -2\eta b 
    \numberthis{}\label{eqn:gradstep}\\
    x^{(k + 1)} &= \proj (y^{(k + 1)}),
    % = \begin{cases} 
    % y^{(k)} & y^{(k)} \in B(R) \\ 
    % R \cdot \frac{y^{(k)}}{\|y^{(k)}\|} & \text{else.}
    %\end{cases}
    \numberthis{}\label{eqn:projectionstep}
\end{align*}
%% or, equivalently,
%% \begin{equation}\label{eqn:argmin-projgrad}
%%   x^{(k+1)} =
%%   \argmin_{x \in \B(R)}
%%   \left(
%%   \frac{1}{2} \|x - (x^{(k)} - \eta \nabla f(x^{(k)}))\|^2
%%   \right),
%% \end{equation}
for nonnegative integer $k$ and step size $\eta$. We make the following assumptions about this procedure. 
\begin{assume}\label{assume:A}
In \eqref{eqn:gradstep}, the 
step size $\eta$ satisfies 
$0 < \eta < \frac{1}{\beta}$.
\end{assume}
\begin{assume}\label{assume:B}
The initial point satisfies $x^{(0)} = 0$.
\end{assume}

\subsection{Asymptotic convergence to a global minimizer}
We begin by providing a few results, which characterize
the iterates of projected gradient descent.
\begin{lem} 
\label{lem:signs}
Let Assumptions \ref{assume:A} and 
\ref{assume:B} hold. Then the iterates of 
gradient descent satisfy 
$(u_i^Tx^{(k)})(u_i^Tb)\leq 0$ for all $i = 1, \dots, n$ and every $k \geq 0$. 
0\end{lem}
\begin{proof}
  Evidently, the claim holds due to Assumption \ref{assume:B} when $k = 0$. Thus, inductively assume that
  for some $k$
  \begin{equation}\label{eqn:inductive-hypothesis}
  (u_i^Tx^{(k)})(u_i^Tb)\leq 0 \qquad  \text{for all $i = 1, \dots, n$.}
  \end{equation}
  By definition, $x^{(k + 1)} = c y^{(k + 1)}$ for some $c \in (0 , 1]$, so it suffices to ensure
    $(u_i^Ty^{(k + 1)})(u_i^Tb) \leq 0$.
  Using \eqref{eqn:inductive-hypothesis} along with Assumption \ref{assume:A},
    \[
    (u_i^Ty^{(k + 1)})(u_i^Tb) = (1 - \eta \lambda_i) (u_i^Tx^{(k)})(u_i^Tb) - \eta (u_i^Tb)^2 \leq 0,
    \]
    since $\eta < \beta^{-1} \leq \lambda_i^{-1}$,
    for all $i =1 ,\dots, n$. This proves the result.  
\end{proof}
The following result shows projected gradient descent is a descent method for \eqref{problem:TR}.
%% \begin{lem}
%% \label{lem:variational-char}
%% Let Assumption \ref{assume:A} hold. Then, for $k > 0$, projected gradient descent iterates obey 
%% \begin{equation}\label{eqn:pgd-var}
%% x^{(k + 1)} = 
%% \argmin_{x \in \B(R)} 
%% \left(\nabla f(x^{(k)})^T(x - x^{(k)}) + \frac{1}{2\eta} \|x - x^{(k)} \|^2\right).
%% \end{equation}
%% \end{lem}
%% \begin{proof}
%%   Basic manipulations imply
%%   \[
%%   \nabla f(x^{(k)})^T(x - x^{(k)}) + \frac{1}{2\eta} \|x - x^{(k)} \|^2 =
%%   \frac{1}{2\eta} \|x - (x^{(k)} - \eta \nabla f(x^{(k)}))\|^2 -\frac{\eta}{2}\|\nabla f(x^{(k)})\|^2. 
%%   \]
%%   Since $\eta > 0$ and $\nabla f(x^{(k)})$ is constant with respect to the minimization in \eqref{eqn:pgd-var},
%%   \begin{equation}
%%   \argmin_{x \in \B(R)}
%%   \left(\nabla f(x^{(k)})^T(x - x^{(k)}) + \frac{1}{2\eta} \|x - x^{(k)} \|^2\right) =
%%   \argmin_{x \in \B(R)}
%%   \left(
%%   \frac{1}{2} \|x - (x^{(k)} - \eta \nabla f(x^{(k)}))\|^2
%%   \right).
%%   \end{equation}
%%   The claim now immediately follows from the formulation of the projected gradient step in \eqref{eqn:argmin-projgrad}.
%% \end{proof}

\begin{lem}
\label{lem:descent-method}
Let Assumption \ref{assume:A} hold. 
Then for any $k > 0$, 
\[
f(x^{(k + 1)}) - 
f(x^{(k)}) \leq 
\left(\frac{\beta}{2} - \frac{1}{2\eta} \right) \|x^{(k + 1)} - x^{(k)}\|^2. 
\]
\end{lem}
\begin{proof}
    Basic manipulations imply
  \[
  \nabla f(x^{(k)})^T(x - x^{(k)}) + \frac{1}{2\eta} \|x - x^{(k)} \|^2 =
  \frac{1}{2\eta} \|x - (x^{(k)} - \eta \nabla f(x^{(k)}))\|^2 -\frac{\eta}{2}\|\nabla f(x^{(k)})\|^2. 
  \]
  Thus, as $\eta > 0$ it follows that
  \[
  \argmin_{x \in \B(R)}
  \left(\nabla f(x^{(k)})^T(x - x^{(k)}) + \frac{1}{2\eta} \|x - x^{(k)} \|^2\right) =
  \argmin_{x \in \B(R)}
  \left(
  \frac{1}{2} \|x - (x^{(k)} - \eta \nabla f(x^{(k)}))\|^2
  \right).
  \]
  Comparing the display above to \eqref{eqn:gradstep},
  \eqref{eqn:projectionstep}, and the definition of $\proj$,
%  \[
  \begin{equation}\label{eqn:PGD-argmin}
  x^{(k + 1)} = \argmin_{x \in \B(R)}
  \left(
  \nabla f(x^{(k)})^T(x - x^{(k)}) + \frac{1}{2\eta} \|x - x^{(k)} \|^2
  \right).
  \end{equation}
%  \]
  Appealing to the $\beta$-smoothness of $f$ and
  evaluating \eqref{eqn:PGD-argmin} at $x^{(k)} \in \B(R)$,
\[
f(x^{(k + 1)}) - f(x^{(k)}) 
\leq 
\nabla f(x^{(k)})^T(x^{(k + 1)} - x^{(k)}) 
+ \frac{\beta}{2} \|x^{(k + 1)} - x^{(k)}\|^2 
\leq 
\left(\frac{\beta}{2} - \frac{1}{2\eta}\right)  \|x^{(k + 1)} - x^{(k)}\|^2.
\]
%% The second inequality follows from noting that plugging $x^{(k)}$ into the
%% argument of $\eqref{eqn:pgd-var}$ yields a value of 0, and therefore plugging in $x^{(k+1)}$
%% must yield a value less than or equal to 0.
\end{proof}
The following result provides 
a useful optimality criterion for 
the trust region subproblem 
\eqref{problem:TR}. 
\begin{thm}[\cite{conn2000}, Corollary 7.2.2.]
\label{thm:optimality}
A point $x \in \B(R)$ is a global minimizer of $f$ subject to $\|x\| \leq R$ if and only if
for some $z \geq 0$, 
\[
(A + zI)x = -b 
\qquad
A  + z I \succeq 0
\qquad 
z(\|x\| - R) = 0.
\]
Furthermore, $x$ is unique if and only if $A + z I \succ 0$. In this case, we write $x = x^\star$. 
\end{thm}
An important special case from Theorem \ref{thm:optimality} is that when $\|x^\star\| < R$, then $\nabla f(x^\star) = 0$.
Furthermore, with a simplifying assumption, we can provide a set of simpler optimality criterion.
\newcommand{\xopt}{\tilde x}
\begin{cor}\label{cor:optimality-criterion2}
  Suppose that $b^Tu_1 \neq 0$. Then if for some $\xopt \in \B(R)$ and $z \geq 0$, it holds that
  \begin{equation}\label{eqn:pgd-optimality}
  (A + zI)\xopt = -b \qquad z(\|\xopt\| - R) = 0 \qquad (u_1^T \xopt)(u_1^T b) \leq 0 
  \end{equation}
  then $\xopt$ is the unique global minimizer to $f$ over $\B(R)$, \ie, $\xopt = x^\star$. 
\end{cor}
\begin{proof}
  Focusing on the first condition, $b^Tu_1 = -(z + \lambda_1)(u_1^T \xopt)$. Thus,
  $b^T u_1 \neq 0$ implies that $(u_1^T\xopt) \neq 0$ and $z + \lambda_1 \neq 0$,
  strengthening the third condition to $(u_1^T \xopt)(u_1^T b) < 0$. But this implies that
  $z + \lambda_1 = -(u_1^Tb)(u_1^T\xopt)/(u_1^T\xopt)^2> 0$,
  which implies that $z > \lambda_i$ for all $i$, whence $A + zI \succ 0$, establishing the result.
\end{proof}

The assumptions along with Corollary \ref{cor:optimality-criterion2}
and Lemmas \ref{lem:signs} and \ref{lem:descent-method}
give us our desired asymptotic convergence gaurantee.
\begin{prop}[Asymptotic convergence]\label{prop:pgd-convergence}
Let Assumptions \ref{assume:A} and 
\ref{assume:B} 
hold, and suppose $b^Tu_1 \neq 0$.
Then as $k \to \infty$, the iterates
of projected gradient descent satisfy $x^{(k)} \to x^\star$
and $f(x^{(k)}) \downarrow f(x^\star)$.
\end{prop}
\begin{proof}
  It suffices to demonstrate that $x^{(k)} \to x^\star$, because then
  the conclusion follows via continuity of $f$ and
  To that end, Lemma \ref{lem:descent-method}.
Lemma \ref{lem:descent-method} and 
Assumption \ref{assume:A} yield the following bound for any integer $T \geq 1$, 
\begin{equation}
\label{eqn:series-convergence}
\left(\frac{1}{2\eta} - \frac{\beta}{2} \right)\sum_{k=0}^{T - 1} 
\|x^{(k + 1)} - x^{(k)}\|^2 
\leq f(x^{(0)}) - f(x^{(T)}) 
\leq f(x^{(0)}) - f^\star. 
\end{equation}
Now, define $\phi: \B(R) \to \R^n$ 
by $\phi(x) = \proj(x - \eta \nabla f(x)) - x$, for points $x \in \B(R)$. 
The bound in \eqref{eqn:series-convergence} implies that the displayed series is convergent as $T \to \infty$ and thus 
$\phi(x^{(k)}) \to 0$.
Note also that the map $\phi$ is evidently continuous, as $\nabla f$ is $\beta$-Lipschitz and $\proj$ is 
non-expansive, thus 1-Lipschitz. 

Suppose now that $\tilde x \in \B(R)$ is a subsequential 
limit of $(x^{(k)})$ (indeed, one exists since this sequence is bounded), and observe by
continuity $\phi(\tilde x) = 0$. To show that $\tilde x = x^\star$, by Corollary \ref{cor:optimality-criterion2}, it
suffices to establish the first two conditions of \eqref{eqn:pgd-optimality}, as the third immediately holds by
Lemma \ref{lem:signs}. Observe first that $\phi(\tilde x) = 0$ implies that for some $c \geq 1$,
\begin{equation}\label{eqn:equality}
\tilde x - \eta \nabla f(\tilde x) = \tilde x - \eta (A \tilde x - b) = c \tilde x.
\end{equation}
Indeed, setting $z = (c - 1)\eta^{-1}$, this implies that $(A + zI)\tilde x = -b$. 
If $\tilde x$ lies on the boundary of $\B(R)$, so that $\|\tilde x\| = R$, then as $z \geq 0$,
this establishes \eqref{eqn:pgd-optimality} and hence $\tilde x = x^\star$. On the other hand,
if $\tilde x$ is in the interior of $\B(R)$, so that $\|\tilde x\| < R$, then
$\phi(\tilde x) = 0$ implies that $c = 1$ in \eqref{eqn:equality}, and thus $z = 0$, once again
establishing \eqref{eqn:pgd-optimality}, hence also that $\tilde x = x^\star$. As this analysis applies
to any such subsequential limit $\tilde x$ of the bounded sequence $(x^{(k)})$, the claim is now proven (since the iterates
lie in $\B(R)$, which is compact).
\end{proof}
We provide some numerical evidence demonstrating the effect of Proposition \ref{prop:pgd-convergence} in
Figure \ref{fig:convergence-2D}.
\begin{figure}
    \centering
    \psfragfig*[width=0.925\linewidth]{figs/convergence}{
    }
    \caption{Three random indefinite instances of the the trust region subproblem \eqref{problem:TR}, with $R = 1$, $\eta = 1/(2\|A\|_{\mathrm{op}})$ and
      $x^{(0)} = 0$. From left to right, the eigenvalues are $\lambda = (-8, 3)$, $\lambda = (-9, 3)$, and $\lambda = (-7,1)$.
      The dots indicate iterates of projected gradient descent and the lines indicate the process $\dot x = -\nabla f(x)$.}
    \label{fig:convergence-2D}
\end{figure}
\section{Non-asymptotic convergence guarantees}
We plan to continue this work by providing convergence rates for this problem. In particular, our current goal  is to obtain a non-asympototic convergence
rate, similar to Theorem 3.1, \cite{carmon2016}.
\bibliography{references}
\end{document}
