\documentclass[11pt]{article}
\usepackage{fullpage,graphicx,psfrag,amsmath,amsfonts,verbatim}
\usepackage[small,bf]{caption}
\usepackage{pstool}
\graphicspath{{figs/}}
\let\epsilon\varepsilon
\input defs.tex
\newcommand{\proj}{\ensuremath{\Pi_{\B(R)}}}
\newcommand\numberthis{\addtocounter{equation}{1}\tag{\theequation}}
\bibliographystyle{alpha}

\title{Projected Gradient Descent Efficiently Solves the Trust Region Subproblem}
\author{Mark Nishimura \and Reese Pathak}

\begin{document}
\maketitle

\begin{abstract}
We show that projected gradient descent asymptotically converges to a global minimizer of 
the trust region subproblem. We then show that iterates shortly hit the boundary, after which
consecutive iterates remain on the boundary. Conditional on a single conjectured inequality
from empirical evidence we are able to show that projected gradient descent achieves
the typical $O(\log(1/\epsilon))$ rate enjoyed by smooth convex functions. 
\end{abstract}

% \newpage
% \tableofcontents
% \newpage

\section{Introduction}
Trust region methods are sequential programming procedures in which heuristics are used to approximately solve a general optimization problem through multiple constrained quadratic programs. As a subroutine, these methods formulate and solve many instances of the following \emph{trust region subproblem}
\begin{equation}\label{problem:TR}
\begin{array}{ll} 
\mbox{minimize} & (1/2)x^TAx + b^T x \\
\mbox{subject to} & 
\|x\| \leq R\\
\end{array}
\end{equation}
with variable $x \in \R^n$. 
The problem data are a symmetric matrix $A \in \R^{n \times n}$, a vector $b \in \R^n$, and a radius parameter $R > 0$. Crucially, the matrix $A$ is possibly indefinite. 
\subsection{Previous works}
The trust region subproblem is well-studied, and thus there many previous works worth mentioning. In earlier papers, the problem was solved either via subspace methods such as Steihaug-Toint (where no global convergence guarantees have been proven, to our knowledge), or using fast eigenvector and eigenvalue computation procedures like the Lanczos method \cite{conn2000, erway2009, gould1999, gould2010}. More recently, however, some authors have provided convergence guarantees for this problem. For example, by reducing the trust region subproblem to a sequence of approximate eigenvector computations, Hazan and Koren \cite{hazan2016}
demonstrate that $\tilde O(1/\sqrt{\epsilon})$\footnote{We use the $\tilde O(\cdot)$ notation to hide logarithmic factors.} matrix-vector multiplies are enough to guarantee an 
$\epsilon$-suboptimal point. In \cite{nguyen2017}, Nguyen and Kilin{\c{c}}{-}Karzan reduce the trust region problem to a convex QCQP using eigenvector calculations, where first-order methods apply. 

However, perhaps the most obvious algorithm to solve \eqref{problem:TR}, is the
\emph{projected gradient method}, which we study in this paper. To our knowledge, the only previous work 
that analyzes the convergence properties of this procedure on 
\eqref{problem:TR} is \cite{tao1998}, where Tao and An augment this procedure by a restarting scheme, requiring 
possibly $O(d)$ restarts, which could scale poorly for large-scale problems. 
We also mention a recent work by Carmon and Duchi \cite{carmon2016}, studying the closely related problem
\begin{equation}\label{problem:QP}
\text{minimize}~ (1/2)x^T A x + b^Tx + (\rho/3) \|x\|_2^3,
\end{equation}
in variable $x \in \R^n$, again with $A$ symmetric, possibly indefinite, and parameter $\rho > 0$. The 
authors analyze gradient descent, proving that $\tilde O(1/\epsilon)$ gradient steps are enough to output an $\epsilon$-suboptimal point.

In this paper we demonstrate that the projected gradient method on \eqref{problem:TR} 
asymptotically converges to a global minimizer on the trust region subproblem.

%% In \S\ref{sec:prelims} 
%% we prove that projected gradient descent is a descent method, in particular, converging to 
%% the global minimizer of the objective in problem \eqref{problem:TR}. 

\subsection{Notation and classical results}
In the sequel, we refer to the objective function as $f: \R^n \to \R$, given by $f(x) = (1/2)x^T Ax + 2b^T x$.
Additionally, the constraint set is the closed ball
$\B(R) \triangleq \{x \in \R^n \mid \|x\| \leq R\}$, 
where $\|\cdot\|$ denotes the Euclidean norm. We use the notation $x^\star$ to denote the global minimum of $f$ when it is unique,
so that $x^\star = \argmin_{x \in \B(R)} f(x)$. We use $f^\star$ to denote the optimal value of $f$, so that
$f^\star = \inf_{x \in \B(R)} f(x)$. Hence, when $x^\star$ exists, $f^\star = f(x^\star)$. 

We fix the eigendecomposition of $A = UDU^T$, where $D = \diag(\lambda_1, \dots, \lambda_n)$, and $U$ has orthonormal
columns $u_i$. We impose without loss that $\lambda_1 \leq \lambda_2 \leq \cdots \leq \lambda_n$. 
By $\|\cdot\|_\mathrm{op}$, we denote the $\ell_2$-operator norm 
$\|M\|_\mathrm{op} = \sup_{\|x\| = 1} 
\|Mx\|$, for any $M \in \R^{n \times n}$. 
A useful identity is that 
$\|M\|_\mathrm{op} = \max_i |\lambda_i(M)|$ when 
$M$ is a symmetric $n \times n$ matrix. We will put $\beta \triangleq \|A\|_{\mathrm{op}}$.

Additionally, say a differentiable function $g: \R^n \to \R$ is 
$L$-smooth on convex set $C \subset \R^n$, provided that 
\[
\|\nabla g(x) - \nabla g(y)\| \leq L \|x - y\| \qquad 
\text{for any $x, y \in C$}.
\]
It is well known that this implies 
\begin{equation}\label{ineq:smoothness}
g(x) - g(y) \leq  \nabla g(y)^T(x - y) + \frac{L}{2}\|x -y\|^2 \qquad 
\text{for any $x, y \in C$}.
\end{equation}
Equivalently, $\|g(x)\|_\mathrm{op} \leq L$, for Lebesgue almost every $x \in C$. For nonempty, closed, convex sets $C 
\subset \R^n$, associate the projection operator $\Pi_C: \R^n \to C$ given by 
\[
\Pi_C(x) = \argmin_{y \in C} \left( \frac12 \|x - y\|^2 \right), 
\]
for any $x \in \R^n$. In the sequel we denote by $I: \R^n \to \R^n$ the identity operator on $\R^n$.

\section{Asymptotic convergence to a global minimizer}\label{sec:prelims}

\subsection{Projected gradient descent}
Projected gradient descent (PGD) begins at an initialization $x^{(0)} \in \R^n$ and generates iterates
\begin{align*}
    y^{(k + 1)} &= x^{(k)} - \eta \nabla f(x^{(k)}) 
    %= (I - 2\eta A)x^{(k-1)} -2\eta b 
    \numberthis{}\label{eqn:gradstep}\\
    x^{(k + 1)} &= \proj (y^{(k + 1)}),
    % = \begin{cases} 
    % y^{(k)} & y^{(k)} \in B(R) \\ 
    % R \cdot \frac{y^{(k)}}{\|y^{(k)}\|} & \text{else.}
    %\end{cases}
    \numberthis{}\label{eqn:projectionstep}
\end{align*}
%% or, equivalently,
%% \begin{equation}\label{eqn:argmin-projgrad}
%%   x^{(k+1)} =
%%   \argmin_{x \in \B(R)}
%%   \left(
%%   \frac{1}{2} \|x - (x^{(k)} - \eta \nabla f(x^{(k)}))\|^2
%%   \right),
%% \end{equation}
for nonnegative integer $k$ and step size $\eta$. We make the following assumptions about this procedure. 
\begin{assume}\label{assume:A}
In \eqref{eqn:gradstep}, the 
step size $\eta$ satisfies 
$0 < \eta < \frac{1}{\beta}$.
\end{assume}
\begin{assume}\label{assume:B}
The initial point satisfies $x^{(0)} = 0$.
\end{assume}

\subsection{Asymptotic convergence to a global minimizer}
We begin by providing a few results, which characterize
the iterates of projected gradient descent.
\begin{lem} 
\label{lem:signs}
Let Assumptions \ref{assume:A} and 
\ref{assume:B} hold. Then the iterates of 
gradient descent satisfy 
$(u_i^Tx^{(k)})(u_i^Tb)\leq 0$ for all $i = 1, \dots, n$ and every $k \geq 0$. 
0\end{lem}
\begin{proof}
  Evidently, the claim holds due to Assumption \ref{assume:B} when $k = 0$. Thus, inductively assume that
  for some $k$
  \begin{equation}\label{eqn:inductive-hypothesis}
  (u_i^Tx^{(k)})(u_i^Tb)\leq 0 \qquad  \text{for all $i = 1, \dots, n$.}
  \end{equation}
  By definition, $x^{(k + 1)} = c y^{(k + 1)}$ for some $c \in (0 , 1]$, so it suffices to ensure
    $(u_i^Ty^{(k + 1)})(u_i^Tb) \leq 0$.
  Using \eqref{eqn:inductive-hypothesis} along with Assumption \ref{assume:A},
    \[
    (u_i^Ty^{(k + 1)})(u_i^Tb) = (1 - \eta \lambda_i) (u_i^Tx^{(k)})(u_i^Tb) - \eta (u_i^Tb)^2 \leq 0,
    \]
    since $\eta < \beta^{-1} \leq \lambda_i^{-1}$,
    for all $i =1 ,\dots, n$. This proves the result.  
\end{proof}
The following result shows projected gradient descent is a descent method for \eqref{problem:TR}.
%% \begin{lem}
%% \label{lem:variational-char}
%% Let Assumption \ref{assume:A} hold. Then, for $k > 0$, projected gradient descent iterates obey 
%% \begin{equation}\label{eqn:pgd-var}
%% x^{(k + 1)} = 
%% \argmin_{x \in \B(R)} 
%% \left(\nabla f(x^{(k)})^T(x - x^{(k)}) + \frac{1}{2\eta} \|x - x^{(k)} \|^2\right).
%% \end{equation}
%% \end{lem}
%% \begin{proof}
%%   Basic manipulations imply
%%   \[
%%   \nabla f(x^{(k)})^T(x - x^{(k)}) + \frac{1}{2\eta} \|x - x^{(k)} \|^2 =
%%   \frac{1}{2\eta} \|x - (x^{(k)} - \eta \nabla f(x^{(k)}))\|^2 -\frac{\eta}{2}\|\nabla f(x^{(k)})\|^2. 
%%   \]
%%   Since $\eta > 0$ and $\nabla f(x^{(k)})$ is constant with respect to the minimization in \eqref{eqn:pgd-var},
%%   \begin{equation}
%%   \argmin_{x \in \B(R)}
%%   \left(\nabla f(x^{(k)})^T(x - x^{(k)}) + \frac{1}{2\eta} \|x - x^{(k)} \|^2\right) =
%%   \argmin_{x \in \B(R)}
%%   \left(
%%   \frac{1}{2} \|x - (x^{(k)} - \eta \nabla f(x^{(k)}))\|^2
%%   \right).
%%   \end{equation}
%%   The claim now immediately follows from the formulation of the projected gradient step in \eqref{eqn:argmin-projgrad}.
%% \end{proof}

\begin{lem}
\label{lem:descent-method}
Let Assumption \ref{assume:A} hold. 
Then for any $k > 0$, 
\[
f(x^{(k + 1)}) - 
f(x^{(k)}) \leq 
\left(\frac{\beta}{2} - \frac{1}{2\eta} \right) \|x^{(k + 1)} - x^{(k)}\|^2. 
\]
\end{lem}
\begin{proof}
    Basic manipulations imply
  \[
  \nabla f(x^{(k)})^T(x - x^{(k)}) + \frac{1}{2\eta} \|x - x^{(k)} \|^2 =
  \frac{1}{2\eta} \|x - (x^{(k)} - \eta \nabla f(x^{(k)}))\|^2 -\frac{\eta}{2}\|\nabla f(x^{(k)})\|^2. 
  \]
  Thus, as $\eta > 0$ it follows that
  \[
  \argmin_{x \in \B(R)}
  \left(\nabla f(x^{(k)})^T(x - x^{(k)}) + \frac{1}{2\eta} \|x - x^{(k)} \|^2\right) =
  \argmin_{x \in \B(R)}
  \left(
  \frac{1}{2} \|x - (x^{(k)} - \eta \nabla f(x^{(k)}))\|^2
  \right).
  \]
  Comparing the display above to \eqref{eqn:gradstep},
  \eqref{eqn:projectionstep}, and the definition of $\proj$,
%  \[
  \begin{equation}\label{eqn:PGD-argmin}
  x^{(k + 1)} = \argmin_{x \in \B(R)}
  \left(
  \nabla f(x^{(k)})^T(x - x^{(k)}) + \frac{1}{2\eta} \|x - x^{(k)} \|^2
  \right).
  \end{equation}
%  \]
  Appealing to the $\beta$-smoothness of $f$ and
  evaluating \eqref{eqn:PGD-argmin} at $x^{(k)} \in \B(R)$,
\[
f(x^{(k + 1)}) - f(x^{(k)}) 
\leq 
\nabla f(x^{(k)})^T(x^{(k + 1)} - x^{(k)}) 
+ \frac{\beta}{2} \|x^{(k + 1)} - x^{(k)}\|^2 
\leq 
\left(\frac{\beta}{2} - \frac{1}{2\eta}\right)  \|x^{(k + 1)} - x^{(k)}\|^2.
\]
%% The second inequality follows from noting that plugging $x^{(k)}$ into the
%% argument of $\eqref{eqn:pgd-var}$ yields a value of 0, and therefore plugging in $x^{(k+1)}$
%% must yield a value less than or equal to 0.
\end{proof}
The following result provides 
a useful optimality criterion for 
the trust region subproblem 
\eqref{problem:TR}. 
\begin{thm}[\cite{conn2000}, Corollary 7.2.2.]
\label{thm:optimality}
A point $x \in \B(R)$ is a global minimizer of $f$ subject to $\|x\| \leq R$ if and only if
for some $z \geq 0$, 
\[
(A + zI)x = -b 
\qquad
A  + z I \succeq 0
\qquad 
z(\|x\| - R) = 0.
\]
Furthermore, $x$ is unique if and only if $A + z I \succ 0$. In this case, we write $x = x^\star$. 
\end{thm}
An important special case from Theorem \ref{thm:optimality} is that when $\|x^\star\| < R$, then $\nabla f(x^\star) = 0$.
Furthermore, with a simplifying assumption, we can provide a set of simpler optimality criterion.
\newcommand{\xopt}{\tilde x}
\begin{cor}\label{cor:optimality-criterion2}
  Suppose that $b^Tu_1 \neq 0$. Then if for some $\xopt \in \B(R)$ and $z \geq 0$, it holds that
  \begin{equation}\label{eqn:pgd-optimality}
  (A + zI)\xopt = -b \qquad z(\|\xopt\| - R) = 0 \qquad (u_1^T \xopt)(u_1^T b) \leq 0 
  \end{equation}
  then $\xopt$ is the unique global minimizer to $f$ over $\B(R)$, \ie, $\xopt = x^\star$. 
\end{cor}
\begin{proof}
  Focusing on the first condition, $b^Tu_1 = -(z + \lambda_1)(u_1^T \xopt)$. Thus,
  $b^T u_1 \neq 0$ implies that $(u_1^T\xopt) \neq 0$ and $z + \lambda_1 \neq 0$,
  strengthening the third condition to $(u_1^T \xopt)(u_1^T b) < 0$. But this implies that
  $z + \lambda_1 = -(u_1^Tb)(u_1^T\xopt)/(u_1^T\xopt)^2> 0$,
  which implies that $z > \lambda_i$ for all $i$, whence $A + zI \succ 0$, establishing the result.
\end{proof}

The assumptions along with Corollary \ref{cor:optimality-criterion2}
and Lemmas \ref{lem:signs} and \ref{lem:descent-method}
give us our desired asymptotic convergence gaurantee.
\begin{prop}[Asymptotic convergence]\label{prop:pgd-convergence}
Let Assumptions \ref{assume:A} and 
\ref{assume:B} 
hold, and suppose $b^Tu_1 \neq 0$.
Then as $k \to \infty$, the iterates
of projected gradient descent satisfy $x^{(k)} \to x^\star$
and $f(x^{(k)}) \downarrow f(x^\star)$.
\end{prop}
\begin{proof}
  It suffices to demonstrate that $x^{(k)} \to x^\star$, because then
  the conclusion follows via continuity of $f$ and
  To that end, Lemma \ref{lem:descent-method}.
Lemma \ref{lem:descent-method} and 
Assumption \ref{assume:A} yield the following bound for any integer $T \geq 1$, 
\begin{equation}
\label{eqn:series-convergence}
\left(\frac{1}{2\eta} - \frac{\beta}{2} \right)\sum_{k=0}^{T - 1} 
\|x^{(k + 1)} - x^{(k)}\|^2 
\leq f(x^{(0)}) - f(x^{(T)}) 
\leq f(x^{(0)}) - f^\star. 
\end{equation}
Now, define $\phi: \B(R) \to \R^n$ 
by $\phi(x) = \proj(x - \eta \nabla f(x)) - x$, for points $x \in \B(R)$. 
The bound in \eqref{eqn:series-convergence} implies that the displayed series is convergent as $T \to \infty$ and thus 
$\phi(x^{(k)}) \to 0$.
Note also that the map $\phi$ is evidently continuous, as $\nabla f$ is $\beta$-Lipschitz and $\proj$ is 
non-expansive, thus 1-Lipschitz. 

Suppose now that $\tilde x \in \B(R)$ is a subsequential 
limit of $(x^{(k)})$ (indeed, one exists since this sequence is bounded), and observe by
continuity $\phi(\tilde x) = 0$. To show that $\tilde x = x^\star$, by Corollary \ref{cor:optimality-criterion2}, it
suffices to establish the first two conditions of \eqref{eqn:pgd-optimality}, as the third immediately holds by
Lemma \ref{lem:signs}. Observe first that $\phi(\tilde x) = 0$ implies that for some $c \geq 1$,
\begin{equation}\label{eqn:equality}
\tilde x - \eta \nabla f(\tilde x) = \tilde x - \eta (A \tilde x - b) = c \tilde x.
\end{equation}
Indeed, setting $z = (c - 1)\eta^{-1}$, this implies that $(A + zI)\tilde x = -b$. 
If $\tilde x$ lies on the boundary of $\B(R)$, so that $\|\tilde x\| = R$, then as $z \geq 0$,
this establishes \eqref{eqn:pgd-optimality} and hence $\tilde x = x^\star$. On the other hand,
if $\tilde x$ is in the interior of $\B(R)$, so that $\|\tilde x\| < R$, then
$\phi(\tilde x) = 0$ implies that $c = 1$ in \eqref{eqn:equality}, and thus $z = 0$, once again
establishing \eqref{eqn:pgd-optimality}, hence also that $\tilde x = x^\star$. As this analysis applies
to any such subsequential limit $\tilde x$ of the bounded sequence $(x^{(k)})$, the claim is now proven (since the iterates
lie in $\B(R)$, which is compact).
\end{proof}
We provide some numerical evidence demonstrating the effect of Proposition \ref{prop:pgd-convergence} in
Figure \ref{fig:convergence-2D}.
\begin{figure}
    \centering
    \psfragfig*[width=0.925\linewidth]{figs/convergence}{
    }
    \caption{Three random indefinite instances of the the trust region subproblem \eqref{problem:TR}, with $R = 1$, $\eta = 1/(2\|A\|_{\mathrm{op}})$ and
      $x^{(0)} = 0$. From left to right, the eigenvalues are $\lambda = (-8, 3)$, $\lambda = (-9, 3)$, and $\lambda = (-7,1)$.
      The dots indicate iterates of projected gradient descent and the lines indicate the process $\dot x = -\nabla f(x)$.}
    \label{fig:convergence-2D}
\end{figure}

\section{Non-asymptotic convergence guarantees}
%% \subsection{Monotone increasing norm}
%% We state and prove a lemma about the norms of projected gradient descent iterates, which will help us bound the time to optimum.
%% \begin{prop}\label{prop:increasing-norm}
%% Let $x^{(0)} = 0$. The iterates of projected gradient descent satisfy
%% \begin{equation} \label{eqn:incr-norm}
%% \|x^{(t+1)}\|_2 \geq \|x^{(t)}\|_2
%% \end{equation}
%% for all $t \geq 0$.
%% \end{prop}
%% Before proving this proposition, we prove the following lemmas about the behavior of PGD with respect to the eigenbasis of the matrix $A$. We use
%% $\mathbf{sgn}(a)$ to denote the sign of a scalar $a$, with $\mathbf{sgn}(0) = 0$.
%% \begin{lem}\label{lem:signs}
%% Let $x^{(0)} = 0$. Then, for $t \geq 1$, the iterates $x^{(t)}$ satisfy
%% \begin{equation}\label{eqn:sign1}
%% \mathbf{sgn}(u_i^Tx^{(t)}) = -\mathbf{sgn}(u_i^Tb).
%% \end{equation}
%% Furthermore, if $\lambda_i \neq 0$, then for $t \geq 0$,
%% \begin{equation}
%% \mathbf{sgn}\left(u_i^Tx^{(t)} + \frac{u_i^Tb}{\lambda_i}\right) = \mathbf{sgn}\left( \frac{u_i^Tb}{\lambda_i}\right).
%% \end{equation}
%% \end{lem}
%% \begin{proof}
%% Since $y^{(1)} = -\eta b$, the first claim is clearly true for $t = 1$. Assume it is true up to $x^{(t-1)}$. Then (noting that the projection operator does not alter sign)
%% \begin{align*}
%% \mathbf{sgn}(u_i^Tx^{(t)}) &= \mathbf{sgn}\left(u_i^T\left[ x^{(t-1)} - \eta \nabla f(x^{(t-1)})\right] \right) \\
%% 					&= \mathbf{sgn}\left( (1 - \eta \lambda_i)u_i^Tx^{(t-1)} - \eta u_i^Tb\right) \\
%% \end{align*}
%% By assumption \ref{assume:A}, $1 - \eta \lambda_i > 0$, so $\mathbf{sgn}\left( (1 - \eta \lambda_i)u_i^Tx^{(t-1)}\right) = \mathbf{sgn}\left(u_i^Tx^{(t-1)}\right)$. Therefore,
%% \[
%% \mathbf{sgn}\left( (1 - \eta \lambda_i)u_i^Tx^{(t-1)} - \eta u_i^Tb\right) = -\mathbf{sgn}(u_i^Tb),
%% \]
%% establishing the first claim.

%% The second claim is also clearly true for $t = 0$. Assume it is true up to $x^{(t-1)}$. Then, for some positive constant $c \leq1$,
%% \begin{align*}
%% \mathbf{sgn}\left(u_i^Tx^{(t)} + \frac{u_i^Tb}{\lambda_i}\right)  
%% 	&= \mathbf{sgn}\left(c u_i^T\left[ x^{(t-1)} - \eta \nabla f(x^{(t-1)}) \right] + \frac{u_i^Tb}{\lambda_i}\right) \\
%% 	&= \mathbf{sgn}\left(c(1 - \eta \lambda_i)u_i^Tx^{(t-1)} + (1 - c\eta \lambda_i) \frac{u_i^Tb}{\lambda_i} \right)
%% \end{align*}

%% By assumption \ref{assume:A}, $(1 - c\eta \lambda_i) > 0$ and $c(1 - \eta \lambda_i) > 0$. If $\lambda_i < 0$, then
%% \[ \mathbf{sgn}\left(\frac{u_i^Tb}{\lambda_i}\right) = - \mathbf{sgn}(u_i^Tb) = \mathbf{sgn}(u_i^Tx^{(t-1)}). \]
%% And therefore, 
%% \begin{align*}
%% \mathbf{sgn}\left(c(1 - \eta \lambda_i)u_i^Tx^{(t-1)} + (1 - c\eta \lambda_i) \frac{u_i^Tb}{\lambda_i} \right)
%% 	&= \mathbf{sgn}\left(u_i^Tx^{(t-1)} + \frac{u_i^Tb}{\lambda_i}\right) \\
%% 	&= \mathbf{sgn}\left(\frac{u_i^Tb}{\lambda_i}\right) 
%% \end{align*}
%% Now assume $\lambda_i > 0$, so that $u_i^Tx^{(t-1)}$ and $\frac{u_i^Tb}{\lambda_i}$ have opposite signs. By inductive assumption,
%% \[ \mathbf{sgn}\left(u_i^Tx^{(t-1)} + \frac{u_i^Tb}{\lambda_i}\right) = \mathbf{sgn}\left( \frac{u_i^Tb}{\lambda_i}\right) \]
%% and so $\left| u_i^Tx^{(t-1)} \right| < \left| \frac{u_i^Tb}{\lambda_i} \right|$.

%%  Since $c \leq 1$ we have that
%% \begin{align*}
%% 0 < c(1 - \eta \lambda_i) &\leq 1 - c \eta \lambda \\
%% \end{align*}
%% so that 
%% \[
%% c(1 - \eta \lambda_i)\left| u_i^Tx^{(t)} \right| < (1 - c \eta \lambda)\left| \frac{u_i^Tb}{\lambda_i} \right|
%% \]
%% which implies
%% \begin{align*}
%% \mathbf{sgn}\left(c(1 - \eta \lambda_i)u_i^Tx^{(t-1)} + (1 - c\eta \lambda_i) \frac{u_i^Tb}{\lambda_i} \right)
%% 	&= \mathbf{sgn}\left((1 - c \eta \lambda)\frac{u_i^Tb}{\lambda_i} \right) \\
%% 	&= \mathbf{sgn}\left(\frac{u_i^Tb}{\lambda_i}\right) 
%% \end{align*}as desired.
%% \end{proof}
%% We are now ready to prove Proposition \ref{prop:increasing-norm}.
%% \begin{proof}
%% Let $t \geq 1$. First, if $\lambda_i = 0$, then $\nabla f(x^{(t)}) = b$, and so $\mathbf{sgn}\left(u_i^T \nabla f(x^{(t)})\right) = \mathbf{sgn}(u_i^Tb)$. 
%% Then if $\lambda_i \neq 0$,
%% \begin{align*}
%% \mathbf{sgn}\left(u_i^T \nabla f(x^{(t)})\right) &= \mathbf{sgn}\left(\lambda_i u_i^Tx^{(t)} + u_i^Tb_i\right) \\
%% 								&=  \mathbf{sgn}(\lambda_i)\mathbf{sgn}\left( u_i^Tx^{(t)} + \frac{u_i^Tb}{\lambda_i}\right) \\
%% 								&= \mathbf{sgn}(\lambda_i)\mathbf{sgn}\left( \frac{u_i^Tb}{\lambda_i} \right) \\
%% 								&= \mathbf{sgn}(u_i^Tb).
%% \end{align*}
%% From this and Lemma \ref{lem:signs}, it follows that
%% \[ (x^{(t)})^T\nabla f(x^{(t)}) = \sum_{i = 1}^n (u_i^Tx^{(t)})( u_i^T\nabla f(x^{(t)})) \leq 0 \]
%% and therefore,
%% \begin{align*}
%% \|x^{(t)} - \eta \nabla f(x^{(t)}) \|^2 &= \|x^{(t)}\|^2 + \eta^2 \|\nabla f(x^{(t)})\|^2 - \eta (x^{(t)})^T\nabla f(x^{(t)}) \\
%% 						&\geq \|x^{(t)}\|^2 \\
%% \end{align*}
%% Let $y^{(t+1)} = x^{(t)} - \eta \nabla f(x^{(t)})$. There are two cases. In the first case, $\|y^{(t+1)}\| > R$, in which case $x^{(t+1)} = \Pi_{\B(R)}(y^{(t+1)})$ and
%% $\|x^{(t+1)}\| = R \geq \| x^{(t)} \|$. In the second case, when $\| y^{(t+1)}\| \leq R$, the projection operator becomes the identity, in which case $y^{(t+1)} = x^{(t+1)}$ and
%% again, $\|x^{(t+1)}\| \geq \|x^{(t)} \|$. 

%% \end{proof}

%% \subsection{Staying on the boundary (TEMPORARY)}
In this section, we use the notation $\pi^{(k)} \in (0, 1]$ to denote
  a constant such that $x^{(k)} = \pi^{(k)} y^{(k)}$. Additionally we tacitly assume that
  Assumptions \ref{assume:A} and \ref{assume:B} hold.

We first prove a technical result about the signs of an iterative process. 
\begin{lem}\label{lem:process-signs}
  Let $\kappa \in \R^n$ satisfy $\kappa_1 \leq \kappa_2 \leq \cdots \leq \kappa_n \leq 1$, and let
  $c^(t)$ denote a non-negative sequence. If $z^{(0)}_i = 0$ for all $i$, and
  \[
  z^{(k + 1)}_i = c^{(k)} (1 - \kappa_i) z^{(k)} + 1
  \]
  for all $k$, then the following three statements hold
  \begin{enumerate}
  \item If $z_j^{(k)} \leq c^{(k - 1)}z_j^{(k - 1)}$ then $z_j^{(k')} \leq c^{(k' - 1)}z_j^{(k' - 1)}$ for all $k ' > k$.
  \item If $z_i^{(k + 1)} \leq c^{(k)}z_i^{(k)}$ then for all $j \geq i$, $z_j^{(k + 1)} \leq c^{(k)}z_j^{(k)}$. 
  \end{enumerate}
  \begin{proof}
    To see (a), it suffices to show (by induction) that the claim holds for $k' = k + 1$. Thus,
    \[
    z_j^{(k + 1)} - c^{(k)} z_j^{(k)} = c^{(k)}(1 - \kappa_j)\left(z_j^{(k)} -c^{(k - 1)}z_j^{(k - 1)}\right) \leq 0,
    \]
    by assumption and $c^{(k)} \geq 0$, $1-\kappa_j\geq 0$.

    To see (b), fix $j \geq i$. Note that evidently $z_i^{(k)} \geq 0$ for all $i, k$ thus
    \[
    \frac{c^{(k)} z_j^{(k)}}{z_j^{(k + 1)}} - \frac{c^{(k)}z_i^{(k)}}{z_i^{(k + 1)}}
    = \frac{(c^{(k)})^2 ( \kappa_j - \kappa_i) z_j^{(k)} z_i^{(k)}}{z_i^{(k + 1)}z_j^{(k +1)}}
    \geq 0.
    \]
    Above, we use that $\kappa_j \geq \kappa_i$ when $j \geq i$. 
    Hence, $c^{(k)}z_j^{(k)}/z_j^{(k + 1)} \geq c^{(k)}z_i^{(k)}/z_i^{(k + 1)} \geq 1$. The claim follows. 
  \end{proof}
\end{lem}  

\begin{lem}\label{lem:inner-prod2}
  Fix $k \in \N$ such that $\nabla f(x^{(k)})^T x^{(k)} \leq 0$. Then ${x^{(k)}}^T A \nabla f(x^{(k)}) \geq \beta {x^{(k)}}^T \nabla f(x^{(k)})$. 
\end{lem}
\begin{proof}
  Note first that $x^{(k')} = \pi^{(k')}y^{(k')}$, thus, for all $k' \leq k$, 
  \[
  \sum_{i=1}^n (u_i^Ty^{(k')})(u_i^T(x^{(k')} - y^{(k' + 1)})) \leq 0.
  \]
  Define the following sets for $k' \leq k$
  \begin{align*}
    I^{(k')}_+ &\triangleq \{i \in [n] : (u_i^Ty^{(k')})(u_i^T(x^{(k')} - y^{(k' + 1)})) \geq 0\}\\
    I^{(k')}_- &\triangleq \{i \in [n] : (u_i^Ty^{(k')})(u_i^T(x^{(k')} - y^{(k' + 1)})) \leq 0\}.
  \end{align*}
  Associated to these sets, define $\lambda_+^{(k')} = \lambda_i$ and $\lambda_{-}^{(k')} = \lambda_j$ for
  $i = \min I^{(k')}_+$, and $j = \max I_-^{(k')}$. 
  Then now observe that, expanding ${y^{(k)}}^T A \nabla f(x^{(k)})$ in the eigenbasis of $A$,
  \begin{align*}
    {y^{(k)}} ^T A \nabla f(x^{(k)}) &= \frac{1}{\eta}\left(
    \sum_{i \in I_+^{(k)}} \lambda_i  (u_i^Ty^{(k)})(u_i^T(x^{(k)} - y^{(k + 1)})) +
    \sum_{i \in I_-^{(k)}} \lambda_i  (u_i^Ty^{(k)})(u_i^T(x^{(k)} - y^{(k + 1)}))\right) \\
    &\geq \frac{1}{\eta}\left(
    \lambda_+^{(k)} \sum_{i \in I_+^{(k)}} (u_i^Ty^{(k)})(u_i^T(x^{(k)} - y^{(k + 1)})) +
    \lambda_-^{(k)} \sum_{i \in I_-^{(k)}} (u_i^Ty^{(k)})(u_i^T(x^{(k)} - y^{(k + 1)}))\right) \\
    &\geq \lambda_-^{(k)}{y^{(k)}}^T\nabla f(x^{(k)}) \geq \beta y^{(k)} \nabla f(x^{(k)}). 
  \end{align*}
  Recalling that $x^{(k)} = \pi^{(k)} y^{(k)}$ with $\pi^{(k)} \in (0, 1]$, this proves the claim.
    The last inequality was obtained by assumption that ${x^{(k)}}^T \nabla f(x^{(k)}) \leq 0$, and that
    $\lambda_i \leq \lambda_n \leq \beta$ for all $i \leq n$. The penultimate inequality is due to $\lambda_-^{(k)} \leq \lambda_+^{(k)}$.
  To see this, it suffices to show that
  \begin{equation}\label{disp:suff-condition-signs}
  (u_i^Ty^{(k)})(u_i^T(x^{(k)} - y^{(k + 1)})) \geq 0
  \quad \text{implies} \quad
  (u_j^Ty^{(k)})(u_j^T(x^{(k)} - y^{(k + 1)})) \geq 0 \quad \text{for all $j \geq i$}
  \end{equation}
  We prove this using Lemma \ref{lem:process-signs}. Indeed, $z^{(k)}_j = (u_j^Ty^{(k)})/(-\eta u_j^Tb)$ for all $k \in \N$. Then
  \[
  z^{(k+1)}_j = \frac{u_j^T((I - \eta A) x^{(k)} - \eta b)}{-\eta u_j^Tb} = \underbrace{\pi^{(k)}}_{\triangleq c^{(k)}}(1 - \underbrace{\eta \lambda_i}_{\kappa_i})
  z_j^{(k)} + 1. 
  \]
  Note that $z^{(0)}_j = 0$, and additionally $\eta \lambda_i = \kappa_i$ is non-decreasing in $i$, and bounded above by $1$ since $\eta \leq 1/\beta$. Additionally,
  by assumption we have $c^{(k')} = \pi^{(k')} = R/\|y^{(k')}\|$, which is evidently non-negative. Thus (2.) of
  Lemma \ref{lem:process-signs} implies that
  \[
  \pi^{(k)}z_i^{(k)} - z_i^{(k + 1)} \geq 0 \quad \text{implies} \quad \pi^{(k)}z_i^{(k)} - z_i^{(k + 1)} \geq 0 \quad \text{for all $j \geq i$.} 
  \]
  Note that as $z_i^{(k)} \geq 0$, this is equivalent to the display in \eqref{disp:suff-condition-signs}. The result is now proven.
\end{proof}


\begin{lem}\label{lem:inner-prod1}
  For all $k \geq 0$, the iterates of projected gradient descent satisfy $b^Tx^{(k)} \leq 0$. 
\end{lem}
\begin{proof}
  We inductively estlabish the following, stronger, result.  
  \begin{equation}\label{eqn:stronger-result}
  \text{for all $i \in [n]$,} \qquad  (u_i^Tx^{(k)})(u_i^T b) \leq 0 \qquad \text{for all $k \in \N$.}
  \end{equation}
  Claim \eqref{eqn:stronger-result} evidently holds when $k = 0$, so now suppose it holds for $k' \leq k$. Fix $i \in [n]$.
  Note
  \begin{align*}
    \sgn(u_i^Tx^{(k + 1)}) &= \sgn\left(\pi^{(k + 1)}((1 - \eta \lambda_i) u_i^Tx^{(k)} - \eta u_i^T b)\right) \\
                                   &= \sgn((1 - \eta \lambda_i) u_i^Tx^{(k)} + \eta (-u_i^T b)) = -\sgn(u_i^Tb)
  \end{align*}
  The final equality holds since $\pi^{(k + 1)} \in (0, 1]$, and the final inequality holds due to the inductive assumption. This proves
    Claim \eqref{eqn:stronger-result}, and the result follows by summing these inequalites:
    $b^Tx^{(k)} = \sum_{i=1}^n ({x^{(k)}}^Tu_i)(u_i^T b) \leq 0$.
\end{proof}
\begin{lem}\label{lem:inner-prod3}
  For all $k \geq 0$, the iterates of projected gradient descent satisfy ${x^{(k)}}^T\nabla f(x^{(k)}) \leq 0$. Furthermore,
  for all $k$, $\|y^{(k + 1)}\| \geq \|x^{(k)}\|$. 
\end{lem}
\begin{proof}
  By definition of the projected gradient descent iteration, we have
  \[
  \|y^{(k + 1)}\|^2 = \|x^{(k)}\|^2 + \eta \|\nabla \grad f(x^{(k)})\|^2 - 2 \eta \grad f(x^{(k)})^T x^{(k)}.
  \]
  Thus, to prove the claim it would be sufficient to show inductively that ${x^{(k)}}^T\nabla f(x^{(k)}) \leq 0$. 
  The basis of induction is clear as the statement trivially holds at $x^{(0)} = 0$.
  Suppose the claim holds for $k$, and note it suffices
  to demonstrate ${y^{(k + 1)}}^T \nabla f(x^{(k + 1)}) \leq 0$. 
  For all $k$, denote $0 < \pi^{(k)} \leq 1$ such that $\pi^{(k)} y^{(k)} = x^{(k)}$. Lemma \ref{lem:inner-prod1} implies
  \begin{align*}
    {y^{(k + 1)}}^T\nabla f(x^{(k+1)}) &\leq 
    \pi^{(k)} \left({x^{(k)}}^T\nabla f(x^{(k)}) - \eta {x^{(k)}}^T A \nabla f(x^{(k)}) - \eta \|\nabla f(x^{(k)})\|^2
    + \eta^2 \nabla f(x^{(k)})^T A \nabla f(x^{(k)}) \right) \\
    &\leq -\pi^{(k)} (\eta-\eta^2 \beta) \|f(x^{(k)})\|^2  + \pi^{(k)}(1 - \eta \beta) {x^{(k)}}^T\nabla f(x^{(k)}) \leq 0 
  \end{align*}
  The penultimate inequality is due Lemma \ref{lem:inner-prod2}, and the final inequality is because $\eta \leq 1/\beta$.
\end{proof}
Lemma \ref{lem:inner-prod3} immediately implies the following claim. 
\begin{cor}\label{cor:stay-bd}
  Suppose an iterate of projected gradient descent satisfies $x^{(\tau)} \in \partial \mathcal{B}(R)$
  for some $\tau \in \N$. Then $x^{(t)} \in \partial \mathcal{B}(R)$ for all $t \geq \tau$.
\end{cor}
\begin{proof}
  Lemma \ref{lem:inner-prod3} implies $\|y^{(\tau + 1)}\|^2 \geq \|x^{(\tau)}\|^2 \geq R^2$, thus
  $x^{(\tau + 1)} \in \partial B(R)$. The claim now follows via induction. 
\end{proof}
In words, once an iterate hits the boundary, all subsequent iterates remain on the boundary.
We believe this result is relevant. Figure \ref{fig:convergence-rates-bd} demonstrates the effect of iterates remaining on the boundary;
empirically we observe exponential convergence, characteristic of gradient descent on smooth, convex functions. 
\begin{figure}[h!t]
    \centering
    \psfragfig*[width=0.5\linewidth]{figs/boundary_vs_interior}{
      \psfrag{kk}{{\small $k$}}
      \psfrag{iiii}{{\scriptsize $\B(R)^o$}}
      \psfrag{bbbb}{{\scriptsize $\partial B(R)$}}
    }
    \caption{Two regimes of convergence, before and after hitting the boundary.}
    \label{fig:convergence-rates-bd}
\end{figure}

The following result bounds the time to the boundary. 
\begin{claim}\label{claim:time-to-bd}
  Suppose that $\lambda_1 < 0$, and $b^Tu_1 \neq 0$. Then after $O\left(\frac{R|\lambda_1|}{|b^Tu_1|\log(1 - \eta \lambda_1)}\right)$ iterations,
  the iterates of projected gradient descent lie on the boundary. 
\end{claim}
\begin{proof}
  Suppose that for $k$ iterations, $\|y^{(k)}\|^2 < R^2$. Then, as $x^{(0)} = 0$, we have
  \[
  y^{(k)} = -\eta \sum_{t=0}^{k-1} (I-\eta A)^t b = -\eta\sum_{i=1}^n \left(\sum_{t=0}^{k-1} (1 - \eta \lambda_i)^t\right) b^Tu_i u_i 
  \]
  Hence,
  \[
  \|y^{(k)}\|^2 = \sum_{i=1}^n \frac{(b^Tu_i)^2}{\lambda_i^2} (1 - (1 - \eta \lambda_i)^k)^2 \geq \frac{(b^Tu_1)^2}{\lambda_1^2}(1 - (1-\eta \lambda_1)^k)^2. 
  \]
  Setting the right-hand side to $R^2$, one obtains that if $k \geq 1 + \frac{R|\lambda_1|}{|b^Tu_1|(\log(1 - \eta \lambda_1)}$, then
  the display above is larger than $R^2$. Thus, the claim now follows via Corollary \ref{cor:stay-bd}.
\end{proof}

\begin{lem}\label{lem:rate}
  Let $\tau^{\mathrm{bd}} = \min \{ k : x^{(k)} \in \partial \mathcal{B}(R)\}$. If for some $\delta  > 0$, and for all
  $k \geq \tau^{\mathrm{bd}}$, $(x^{(k+1)} - \eta \delta x^{(k)})^Tx^\star \geq (1 - \eta \delta)R^2$, then $f(x^{(t)})$ is $\epsilon > 0$-suboptimal
  provided that $t \geq \tau^{\mathrm{bd}} + \frac{1}{\eta \delta} \log\left(\frac{2 R^2 (\beta + z)}{\epsilon}\right)$. 
\end{lem}
\begin{proof}
  By hypothesis, we have $\|x^{(k + 1)} - x^\star\|^2 \leq (1 - \eta \delta)^2 \|x^{(k)} - x^\star\|^2$, provided
  $k \geq \tau^{\mathrm{bd}}$. Thus, as $\delta < 1 /\eta$, and with $\tau := \tau^{\mathrm{bd}}$,
  \[
  \|x^{(k + \tau)} - x^\star\|^2 \leq (1 - \eta \delta)^{2k} \|x^{\tau} - x^\star\|^2 \leq 4R^2 (1 - \eta \delta)^k \leq 4R^2 e^{-2\eta \delta k},
  \]
  as $1 + \alpha \leq e^{\alpha}$ for all $\alpha \in \R$. Now appealing to the $\beta$-smoothness of $f$,
  \[ 
  f(x^{(k + \tau)}) - f^\star \leq 2(\beta + z) R^2 e^{-\eta \delta k} \leq \epsilon,
  \]
  provided that $k \geq \frac{1}{\eta \delta} \log((2 R^2(\beta + z))/\epsilon)$. Hence if
  \[
  t \geq \tau^{\mathrm{bd}} + \frac{1}{\eta \delta} \log\left(\frac{2 R^2 (\beta + z)}{\epsilon}\right),
  \]
  then $f(x^{(t)}) - f^\star \leq \epsilon$. 
\end{proof}
\begin{conj}\label{conj:rate}
  In the notation of Lemma \ref{lem:rate}, for all $k \geq \tau^{\mathrm{bd}}$,
  $(x^{(k+1)} - \eta \delta x^{(k)})^Tx^\star \geq (1 - \eta \delta)R^2$ holds with $\delta = 1/(z + \lambda_1)$. 
\end{conj}
Conditional on Conjecture \ref{conj:rate}, this implies the $\tilde O(\log(1/\epsilon))$ convergence rate enjoyed
by smooth, convex functions under the gradient descent iteration. Of course, the most obvious open problem is to prove (or disprove)
Conjecture \ref{conj:rate}. 
\bibliography{references}
\end{document}
