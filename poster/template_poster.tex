% Template file for an a0 landscape poster.
% Written by Graeme, 2001-03 based on Norman's original microlensing
% poster.
%
% See discussion and documentation at
% <http://www.astro.gla.ac.uk/users/norman/docs/posters/> 
%
% $Id: poster-template-landscape.tex,v 1.2 2002/12/03 11:25:46 norman Exp $


% Default mode is landscape, which is what we want, however dvips and
% a0poster do not quite do the right thing, so we end up with text in
% landscape style (wide and short) down a portrait page (narrow and
% long). Printing this onto the a0 printer chops the right hand edge.
% However, 'psnup' can save the day, reorienting the text so that the
% poster prints lengthways down an a0 portrait bounding box.
%
% 'psnup -w85cm -h119cm -f poster_from_dvips.ps poster_in_landscape.ps'

\documentclass[a0]{a0poster}
% You might find the 'draft' option to a0 poster useful if you have
% lots of graphics, because they can take some time to process and
% display. (\documentclass[a0,draft]{a0poster})
\input defs
\pagestyle{empty}
\usepackage{etoolbox}
\patchcmd{\thebibliography}{\section*{\refname}}{}{}{}
\setcounter{secnumdepth}{0}
\renewcommand{\familydefault}{\sfdefault}
\newcommand{\QED}{~~\rule[-1pt]{8pt}{8pt}}\def\qed{\QED}
\usepackage{amsthm, amsmath, amssymb, pstool}
%% STYLES
\newtheorem{thm}{Theorem}
\newtheorem{prop}[thm]{Proposition}
\newtheorem{lem}[thm]{Lemma}
\newtheorem{cor}[thm]{Corollary}
\newtheorem{fact}[thm]{Fact}
\theoremstyle{definition}
\newtheorem{defn}[thm]{Definition}
\newtheorem{assume}[thm]{Assumption}
\newtheorem{example}[thm]{Example}
\bibliographystyle{alpha}

\renewcommand{\reals}{{\mbox{\bf R}}}
\newcommand{\B}{\mathcal{B}}
% The textpos package is necessary to position textblocks at arbitary 
% places on the page.
\usepackage[absolute]{textpos}

\usepackage{fleqn,psfrag,wrapfig,tikz}

\usepackage[papersize={38in,28in}]{geometry}

% Graphics to include graphics. Times is nice on posters, but you
% might want to switch it off and go for CMR fonts.
\usepackage{graphicx}


% we are running pdflatex, so convert .eps files to .pdf
%\usepackage[pdftex]{graphicx}
%\usepackage{epstopdf}

% These colours are tried and tested for titles and headers. Don't
% over use color!
\usepackage{color}
\definecolor{Red}{rgb}{0.9,0.0,0.1}

\definecolor{bluegray}{rgb}{0.15,0.20,0.40}
\definecolor{bluegraylight}{rgb}{0.35,0.40,0.60}
\definecolor{gray}{rgb}{0.3,0.3,0.3}
\definecolor{lightgray}{rgb}{0.7,0.7,0.7}
\definecolor{darkblue}{rgb}{0.2,0.2,1.0}
\definecolor{darkgreen}{rgb}{0.0,0.5,0.3}

\renewcommand{\labelitemi}{\textcolor{bluegray}\textbullet}
\renewcommand{\labelitemii}{\textcolor{bluegray}{--}}

\setlength{\labelsep}{0.5em}


% see documentation for a0poster class for the size options here
\let\Textsize\normalsize
%\def\Head#1{\noindent\hbox to \hsize{\hfil{\LARGE\color{bluegray} #1}}\bigskip}
\def\Head#1{\noindent{\LARGE\color{bluegray} #1}\bigskip}
\def\LHead#1{\noindent{\LARGE\color{bluegray} #1}\bigskip}
\def\Subhead#1{\noindent{\large\color{bluegray} #1}\bigskip}
\def\Title#1{\noindent{\VeryHuge\color{Red} #1}}

% Set up the grid
%
% Note that [40mm,40mm] is the margin round the edge of the page --
% it is _not_ the grid size. That is always defined as 
% PAGE_WIDTH/HGRID and PAGE_HEIGHT/VGRID. In this case we use
% 23 x 12. This gives us three columns of width 7 boxes, with a gap of
% width 1 in between them. 12 vertical boxes is a good number to work
% with.
%
% Note however that texblocks can be positioned fractionally as well,
% so really any convenient grid size can be used.
%
\TPGrid[40mm,40mm]{23}{12}      % 3 cols of width 7, plus 2 gaps width 1

\parindent=0pt
\parskip=0.2\baselineskip
\usepackage[small,bf]{caption}

\begin{document}

% Understanding textblocks is the key to being able to do a poster in
% LaTeX. In
%
%    \begin{textblock}{wid}(x,y)
%    ...
%    \end{textblock}
%
% the first argument gives the block width in units of the grid
% cells specified above in \TPGrid; the second gives the (x,y)
% position on the grid, with the y axis pointing down.

% You will have to do a lot of previewing to get everything in the 
% right place.

% This gives good title positioning for a portrait poster.
% Watch out for hyphenation in titles - LaTeX will do it
% but it looks awful.
\begin{textblock}{23}(0,0)
\Title{Projected Gradient Descent Solves the Trust Region Problem}
\end{textblock}

\begin{textblock}{23}(0,0.6)
{
\LARGE
Mark Nishimura, Reese Pathak
}

{
\Large
\color{bluegray}
\emph{EE364b, Convex Optimization II Project}
}
\end{textblock}


% Uni logo in the top right corner. A&A in the bottom left. Gives a
% good visual balance, but you may want to change this depending upon
% the graphics that are in your poster.
%\begin{textblock}{2}(0,10)
%Your logo here
%%\includegraphics{/usr/local/share/images/AandA.epsf}
%\end{textblock}

%\begin{textblock}{2}(21.2,0)
%Another logo here
%%\resizebox{2\TPHorizModule}{!}{\includegraphics{/usr/local/share/images/GUVIu/GUVIu.eps}}
%\end{textblock}


\begin{textblock}{7.0}(0,1.5)

\hrule\medskip
\Head{Introduction}\\
Trust region methods are sequential programming procedures which formulate and solve many instances of the following \textbf{trust region problem}
\begin{equation}\label{problem:TR}
\begin{array}{ll} 
\mbox{minimize} & (1/2)x^TAx + b^T x \\
\mbox{subject to} & 
\|x\| \leq R\\
\end{array}
\end{equation}
with variable $x$. Do \textbf{not} assume $A$ is definite.

\textbf{Recall...}
If $A \in \reals^{n \times n}$ is symmetric then for $\Lambda$, orthonormal $U$,
\[
A = U\Lambda U^T \quad \Lambda = \diag(\lambda) \quad
\lambda_1 \leq \cdots \leq \lambda_n \quad
U = [u_1 \mid \cdots \mid u_n]
\]
Also have for $f: \reals^n \to \reals$ with $L$-Lipschitz gradient
\[
f(x) - f(y) \leq  \nabla f(y)^T(x - y) + \frac{L}{2}\|x -y\|^2
\]
\medskip
\hrule\medskip
\Head{Projected Gradient Descent}\\
We investigate the behavior of \textbf{projected gradient descent} (PGD)
which begins at an initialization $x^{(0)} \in \reals^n$ and generates iterates
\[
\begin{array}{ll}
    y^{(k + 1)} &= x^{(k)} - \eta \nabla f(x^{(k)})  \\
    x^{(k + 1)} &= \Pi_{\mathcal{B}(R)} (y^{(k + 1)}). 
\end{array}
\]
We make the following assumptions about this procedure:
\begin{itemize}
\item step size satisfies $0 < \eta < 1/\|A\|_\mathrm{op}$
\item initialize at $x^{(0)} = 0 \in \reals^n$. 
\end{itemize}
\medskip
\hrule\medskip
\Head{Variational interpretation}\\
Complete the square to verify that PGD iterates satisfy (cf. Nesterov)
\[
x^{(k + 1)} = 
\argmin_{x \in \mathcal{B}(R)} 
\left(\nabla f(x^{(k)})^T(x - x^{(k)}) + \frac{1}{2\eta} \|x - x^{(k)} \|^2\right).
\]
(Essentially) immediately implies that PGD is a descent method:
\[
f(x^{(k + 1)}) - 
f(x^{(k)}) \leq 
\left(\frac{\beta}{2} - \frac{1}{2\eta} \right) \|x^{(k + 1)} - x^{(k)}\|^2. 
\]
\medskip
\hrule\medskip
\Head{Optimality criterion}\\
The following result provides 
a useful optimality condition.
\begin{thm}[\cite{conn2000}, Corollary 7.2.2.]
\label{thm:optimality}
A point $x \in \B(R)$ is a global minimizer of $f$ subject to $\|x\| \leq R$ if and only if
for some $z \geq 0$, 
\[
(A + zI)x = -b 
\qquad
A  + z I \succeq 0
\qquad 
z(\|x\| - R) = 0.
\]
Furthermore, $x$ is unique iff $A + z I \succ 0$. In this case, we write $x = x^\star$. 
\end{thm}

We show (and make use of) the following weaker statement. 
\newcommand{\xopt}{\tilde x}
\begin{cor}\label{cor:optimality-criterion2}
  Suppose that $b^Tu_1 \neq 0$. Then if at $\xopt \in \B(R)$,  $z \geq 0$, have
  \[
  (A + zI)\xopt = -b \qquad z(\|\xopt\| - R) = 0 \qquad (u_1^T \xopt)(u_1^T b) \leq 0
  \]
  then $\xopt$
  is the unique global minimizer to $f$ over $\B(R)$, \ie, $\xopt = x^\star$. 
\end{cor}
\end{textblock}

\begin{textblock}{7.0}(8,1.5)
\medskip
\hrule\medskip
\Head{Asymptotic result}\\
Under mild assumptions we obtain the following result.
\begin{prop}[Asymptotic convergence]
  Let the step-size and initialization assumptions hold,
  and assume further that suppose $b^Tu_1 \neq 0$.
  Then as $k \to \infty$, the iterates
  of projected gradient descent satisfy $x^{(k)} \to x^\star$
  and $f(x^{(k)}) \downarrow f(x^\star)$, where $x^\star$ is the unique
  global minimizer to $f$ over $\mathcal{B}(R)$. 
\end{prop}
\begin{itemize}
\item
  In English, unless you have a pretty sick trust region problem, PGD eventually
  gets to the global minimizer of $f$.
\item \emph{Disclaimer:} this statement (nor its proof)
  admit an obvious convergence rate. This means, you could get to opt
  quite slowly (though not in practice, \dots)
\end{itemize}
\medskip
\hrule\medskip
\Head{Convergence proof}\\
\textbf{Idea:} Show that $\|x^{(k + 1)} - x^{(k)}\|_2^2 \to 0$ and
hope that's enough.
\begin{itemize}
\item Use descent method inequality to show that
  \begin{equation}\label{eq:descent}
  x^{(k + 1)} - x^{(k)} \to \infty.
  \end{equation}
\item Introduce continuous map $g: \B(R) \to \reals^n$,
  \[
  x \mapsto \Pi_{\B(R)}(x - \eta \nabla f(x)) - x
  \]
  \begin{itemize}
  \item this is just a single PGD step
  \item can rewrite Eq. \eqref{eq:descent} as $g(x^{(k)}) \to 0$
  \end{itemize}
\item Consider a subsequential limit (one exists since
  $x^{(k)}$ lie in compact set) $L$
  \begin{itemize}
  \item Use continuity to conclude that limits satisfy
    $g(L) = 0$
  \item Use the optimality criterion, projection map, and case work
    to analyze $\ker g$
  \item Conclude that $L = x^\star$
  \end{itemize}
\item The analysis above applies to any limit point, so we're done (here we use
  the fact that $\{x^{(k)}\} \subset K = \B(R)$) 
\end{itemize}
\medskip
\hrule\medskip
\Head{Numerical example}\\
Example below has $A \in \mathbf{R}^{2 \times 2}$ with
$\lambda_1(A) = -8$, $\lambda_2(A) = 3$. We took $R=1$, $\eta = 1/16$.
Note that $b^Tu_1 = 0.790857$ and $\|x^\star\| = 1$. 
\begin{figure}
  \centering
  \psfragfig*[width=0.30\linewidth]{figs/convergence_poster}{
  }
  \caption{Trust region problem. Black diamonds are $x^{(k)}$, red dot is $x^\star$, blue circle is $\partial \B(R)$.}
\end{figure}
\end{textblock}

\begin{textblock}{7.0}(16,1.5)
\medskip
\hrule\medskip
\Head{Ideas for non-asymptotics}\\
Basically, it looks like a non-asymptotic proof of convergence could follow
the following lines (roughly)
\begin{enumerate}
\item {\color{darkgreen}{Show that there is a $\tau^{\mathrm{bd}}$ such that
  when $t \geq \tau^{\mathrm{bd}}$, we can guarantee that you've used projection
  at least once (\ie, you've hit the boundary)}}
  \[
 \|y^{(t)}\|^2 = \eta^2 \left\|\sum_{k=0}^{t - 1} (I - \eta A)b \right\|^2
 = \sum_{i = 1}^n \left(\frac{b^Tu_i}{\lambda_i}\right)^2
 (1 - (1 - \eta \lambda_i)^{t})^2 > R^2
 \]
\item {\color{gray}Show that once the boundary is reached, successive iterates remain on the
  boundary. In other words, $\|y^{(t + 1)}\| > R$ for all
  $t \geq \tau^{\mathrm{bd}}$.
  \begin{figure}
    \centering
    \psfragfig*[width=0.4\linewidth]{figs/convergence_poster2}
    \caption{Initializing randomly on the boundary of $\B(R)$ doesn't always work!}
  \end{figure}
  \begin{itemize}
  \item Importantly, property of remaining on boundary for all successive iterates is not true of
    all points $x \in \partial \B(R)$, \dots
  \end{itemize}
\item Show a contraction inequality like (for $k \geq \tau^{\mathrm{bd}}$)
  \[
  \|x^{(k + 1)} - x^\star\| \leq (1 - \epsilon) \|x^{(k)} - x^{\star}\|
  \qquad (\epsilon > 0)
  \]
  \begin{figure}
    \centering
    \psfragfig*[width=0.3\linewidth]{figs/boundary_vs_interior}{
      \psfrag{kk}{{\small $k$}}
      \psfrag{iiii}{{\scriptsize $\B(R)^o$}}
      \psfrag{bbbb}{{\scriptsize $\partial B(R)$}}
    }
    \caption{Two regimes of convergence}
  \end{figure}
\item Conclude via smoothness, standard GD analysis for smooth problems.}
\end{enumerate}

\medskip
\hrule\medskip \nocite{carmon2016}
\Head{References \& Acknowledgements} \\
\vspace{-15mm}
{\footnotesize \bibliography{references}}
Thanks to Yair and John for putting up with our (mostly stupid) questions.
\end{textblock}
\end{document}
